\subsection{Features}\label{Features}

The programs provided by the \gls{lfl} system exhibit a smaller set of features than their standard equivalents.

\subsubsection{Compliant}

Programs originating from inherited standard packages and those from selected alternative packages including mawk, dash, less, mandoc as well as a subset of programs provided by sbase are \gls{lsb} compliant.

\begin{itemize}
    \item mawk - awk
    \item dash - sh
    \item less - less (\enquote{more} emulation)
    \item mandoc - man
    \item sbase - basename, cat, chgrp, chmod, chown, cksum, cmp, comm, cut, date, dd, dirname, du, echo, env, expand, expr, false, find, fold, grep, head, hostname, join, kill, ln, logger, logname, md5sum, mkdir, mkfifo, mknod, mktemp, nice, nl, nohup, paste, pathchk, printf, pwd, renice, rmdir, sed, seq, sleep, sort, split, strings, sync, tail, tee, test / [, time, touch, tr, true, tsort, tty, uname, unexpand, uniq, wc
\end{itemize}

\subsubsection{Non-compliant}

The programs provided by ubase as well as a subset of programs provided by sbase are non-compliant with the \gls{lsb} specification. Their implementations are missing certain features, nonetheless they exhibit a high degree of compatability with standard implementations.

% UBASE

\begin{table}[!ht]
    \centering
    \begin{tabular}{|c|c|c|}
        \hline
        Program & Option/s & Description or discrepancy \\
        \hline
        \hline
        %%%%%%%%%%%%%%%%%%%
        \multirow{3}{*}{dd}
        %%%%%%%%%%%%%%%%%%%
        & ibs & Specify the input block size. \\
        \cline{2-3}
        & obs & Specify the output block size. \\
        \cline{2-3}
        & cbs & Specify the conversion block size. \\
        \hline
        \hline
        %%%%%%%%%%%%%%%%%%%
        \multirow{4}{*}{df}
        %%%%%%%%%%%%%%%%%%%
        & \multirow{2}{*}{-P} & \multirow{2}{*}{
            \begin{tabular}{c}
                Produce output in the format described \\
                in the \glsxtrshort{stdout} section.
            \end{tabular}
        } \\
        & & \\
        \cline{2-3}
        & \multirow{2}{*}{-t} & \multirow{2}{*}{
            \begin{tabular}{c}
                Include total allocated-space figures \\
                in the output.
            \end{tabular}
        } \\
        & & \\
        \hline
        \hline
        %%%%%%%%%%%%%%%%%%%%%%
        \multirow{2}{*}{dmesg}
        %%%%%%%%%%%%%%%%%%%%%%
        & \multirow{2}{*}{-s} & \multirow{2}{*}{
            \begin{tabular}{c}
                Use a buffer of bufsize to query the system \\
                message buffer. This is 16392 by default.
            \end{tabular}
        } \\
        & & \\
        \hline
        \hline
        %%%%%%%%%%%%%%%%%%%
        \multirow{1}{*}{id}
        %%%%%%%%%%%%%%%%%%%
        & -r & Output the real \glsxtrshort{id} instead of the effective \glsxtrshort{id}. \\
        \hline
        \hline
        \multicolumn{3}{|c|}{\dots} \\
        \hline
    \end{tabular}
\end{table}

\begin{table}[!ht]
    \centering
    \begin{tabular}{|c|c|c|}
        \hline
        Program & Option/s & Description or discrepancy \\
        \hline
        \hline
        %%%%%%%%%%%%%%%%%%%%%%%%%
        \multirow{8}{*}{killall5}
        %%%%%%%%%%%%%%%%%%%%%%%%%
        & \multirow{1}{*}{-e} & \multirow{1}{*}{
            \begin{tabular}{c}
                Require an exact match for very long names.
            \end{tabular}
        } \\
        \cline{2-3}
        & \multirow{2}{*}{-g} & \multirow{2}{*}{
            \begin{tabular}{c}
                Kill the process group to which \\
                the process belongs.
            \end{tabular}
        } \\
        & & \\
        \cline{2-3}
        & \multirow{1}{*}{-i} & \multirow{1}{*}{
            \begin{tabular}{c}
                Ask for confirmation before killing process.
            \end{tabular}
        } \\
        \cline{2-3}
        & \multirow{1}{*}{-l} & \multirow{1}{*}{
            \begin{tabular}{c}
                List all known signal names.
            \end{tabular}
        } \\
        \cline{2-3}
        & \multirow{1}{*}{-q} & \multirow{1}{*}{
            \begin{tabular}{c}
                Silent mode.
            \end{tabular}
        } \\
        \cline{2-3}
        & \multirow{1}{*}{-V} & \multirow{1}{*}{
            \begin{tabular}{c}
                Display version information.
            \end{tabular}
        } \\
        \cline{2-3}
        & \multirow{1}{*}{-v} & \multirow{1}{*}{
            \begin{tabular}{c}
                Report if signal was successfully sent.
            \end{tabular}
        } \\
        \hline
        \hline
        %%%%%%%%%%%%%%%%%%%%%%
        \multirow{1}{*}{mknod}
        %%%%%%%%%%%%%%%%%%%%%%
        & \multirow{1}{*}{p} & \multirow{1}{*}{
            \begin{tabular}{c}
                Create a \glsxtrshort{fifo}.
            \end{tabular}
        } \\
        \hline
        \hline
        %%%%%%%%%%%%%%%%%%%%%%%
        \multirow{13}{*}{mount}
        %%%%%%%%%%%%%%%%%%%%%%%
        & \multirow{2}{*}{-F} & \multirow{2}{*}{
            \begin{tabular}{c}
                Fork a new incarnation of mount for \\
                each device to be mounted.
            \end{tabular}
        } \\
        & & \\
        \cline{2-3}
        & \multirow{1}{*}{-f} & \multirow{1}{*}{
            \begin{tabular}{c}
                Dry run mode.
            \end{tabular}
        } \\
        \cline{2-3}
        & \multirow{2}{*}{-L} & \multirow{2}{*}{
            \begin{tabular}{c}
                Mount the partition that has the \\
                specified label.
            \end{tabular}
        } \\
        & & \\
        \cline{2-3}
        & \multirow{1}{*}{-r} & \multirow{1}{*}{
            \begin{tabular}{c}
                Mount the file system read-only.
            \end{tabular}
        } \\
        \cline{2-3}
        & \multirow{2}{*}{-s} & \multirow{2}{*}{
            \begin{tabular}{c}
                Ignore mount options not supported \\
                by a file system type.
            \end{tabular}
        } \\
        & & \\
        \cline{2-3}
        & \multirow{2}{*}{-U} & \multirow{2}{*}{
            \begin{tabular}{c}
                Mount the partition that has the \\
                specified uuid.
            \end{tabular}
        } \\
        & & \\
        \cline{2-3}
        & \multirow{1}{*}{-V} & \multirow{1}{*}{
            \begin{tabular}{c}
                Display version information.
            \end{tabular}
        } \\
        \cline{2-3}
        & \multirow{1}{*}{-v} & \multirow{1}{*}{
            \begin{tabular}{c}
                Invoke verbose mode.
            \end{tabular}
        } \\
        \cline{2-3}
        & \multirow{1}{*}{-w} & \multirow{1}{*}{
            \begin{tabular}{c}
                Mount the file system read/write.
            \end{tabular}
        } \\
        \hline
        \hline
        %%%%%%%%%%%%%%%%%%%%%%%%
        \multirow{15}{*}{passwd}
        %%%%%%%%%%%%%%%%%%%%%%%%
        & \multirow{3}{*}{-i} & \multirow{3}{*}{
            \begin{tabular}{c}
                Disable an account after the \\
                password has been expired for \\
                the given number of days.
            \end{tabular}
        } \\
        & & \\
        & & \\
        \cline{2-3}
        & \multirow{3}{*}{-l} & \multirow{3}{*}{
            \begin{tabular}{c}
                Disable an account by changing the \\
                password to a value which matches \\
                no possible encrypted value.
            \end{tabular}
        } \\
        & & \\
        & & \\
        \cline{2-3}
        & \multirow{2}{*}{-n} & \multirow{2}{*}{
            \begin{tabular}{c}
                Set the minimum number of days \\
                before a password may be changed.
            \end{tabular}
        } \\
        & & \\
        \cline{2-3}
        & \multirow{2}{*}{-u} & \multirow{2}{*}{
            \begin{tabular}{c}
                Re-enable an account by changing \\
                the password back to its previous value.
            \end{tabular}
        } \\
        & & \\
        \cline{2-3}
        & \multirow{3}{*}{-w} & \multirow{3}{*}{
            \begin{tabular}{c}
                Set the number of days warning \\
                the user will receive before their \\
                password will expire.
            \end{tabular}
        } \\
        & & \\
        & & \\
        \cline{2-3}
        & \multirow{2}{*}{-x} & \multirow{2}{*}{
            \begin{tabular}{c}
                Set the maximum number of days \\
                a password remains valid.
            \end{tabular}
        } \\
        & & \\
        \hline
        \hline
        \multicolumn{3}{|c|}{\dots} \\
        \hline
    \end{tabular}
\end{table}


\begin{table}[!ht]
    \centering
    \begin{tabular}{|c|c|c|}
        \hline
        Program & Option/s & Description or discrepancy \\
        \hline
        \hline
        %%%%%%%%%%%%%%%%%%%%%%
        \multirow{2}{*}{pidof}
        %%%%%%%%%%%%%%%%%%%%%%
        & \multirow{2}{*}{-x} & \multirow{2}{*}{
            \begin{tabular}{c}
                Also return process id's of shells \\
                running the named scripts.
            \end{tabular}
        } \\
        & & \\
        \hline
        \hline
        %%%%%%%%%%%%%%%%%%%%
        \multirow{19}{*}{ps}
        %%%%%%%%%%%%%%%%%%%%
        & \multirow{2}{*}{-G} & \multirow{2}{*}{
            \begin{tabular}{c}
                Write information for processes whose \\
                real group \glsxtrshort{id} numbers are given in grouplist.
            \end{tabular}
        } \\
        & & \\
        \cline{2-3}
        & \multirow{2}{*}{-g} & \multirow{2}{*}{
            \begin{tabular}{c}
                Write information for processes whose \\
                session leaders are given in grouplist.
            \end{tabular}
        } \\
        & & \\
        \cline{2-3}
        & \multirow{1}{*}{-l} & \multirow{1}{*}{
            \begin{tabular}{c}
                Generate a long listing.
            \end{tabular}
        } \\
        \cline{2-3}
        & \multirow{2}{*}{-n} & \multirow{2}{*}{
            \begin{tabular}{c}
                Specify the name of an alternative system \\
                namelist file in place of the default.
            \end{tabular}
        } \\
        & & \\
        \cline{2-3}
        & \multirow{2}{*}{-o} & \multirow{2}{*}{
            \begin{tabular}{c}
                Write information according to the \\
                format specification given in format.
            \end{tabular}
        } \\
        & & \\
        \cline{2-3}
        & \multirow{2}{*}{-p} & \multirow{2}{*}{
            \begin{tabular}{c}
                Write information for processes whose \\
                process \glsxtrshort{id} numbers are given in proclist.
            \end{tabular}
        } \\
        & & \\
        \cline{2-3}
        & \multirow{2}{*}{-t} & \multirow{2}{*}{
            \begin{tabular}{c}
                Write information for processes associated \\
                with terminals given in termlist.
            \end{tabular}
        } \\
        & & \\
        \cline{2-3}
        & \multirow{3}{*}{-U} & \multirow{3}{*}{
            \begin{tabular}{c}
                Write information for processes whose \\
                real user \glsxtrshort{id} numbers or login names \\
                are given in userlist.
            \end{tabular}
        } \\
        & & \\
        & & \\
        \cline{2-3}
        & \multirow{3}{*}{-u} & \multirow{3}{*}{
            \begin{tabular}{c}
                Write information for processes whose \\
                user \glsxtrshort{id} numbers or login names are \\
                given in userlist.
            \end{tabular}
        } \\
        & & \\
        & & \\
        \hline
        \hline
        %%%%%%%%%%%%%%%%%%%%
        \multirow{3}{*}{su}
        %%%%%%%%%%%%%%%%%%%%
        & \multirow{1}{*}{-c} & \multirow{1}{*}{
            \begin{tabular}{c}
                Invoke the shell with the option -c command.
            \end{tabular}
        } \\
        \cline{2-3}
        & \multirow{1}{*}{-m} & \multirow{1}{*}{
            \begin{tabular}{c}
                Pass the current environment to the invoked shell.
            \end{tabular}
        } \\
        \cline{2-3}
        & \multirow{1}{*}{-s} & \multirow{1}{*}{
            \begin{tabular}{c}
                Invoke shell as the command interpreter.
            \end{tabular}
        } \\
        \hline
        \hline
        %%%%%%%%%%%%%%%%%%%%%%%
        \multirow{5}{*}{umount}
        %%%%%%%%%%%%%%%%%%%%%%%
        & \multirow{1}{*}{-r} & \multirow{1}{*}{
            \begin{tabular}{c}
                Try to remount read-only if unmounting fails.
            \end{tabular}
        } \\
        \cline{2-3}
        & \multirow{2}{*}{-t} & \multirow{2}{*}{
            \begin{tabular}{c}
                Indicate that the actions should only \\
                be taken on file systems of the specified type.
            \end{tabular}
        } \\
        & & \\
        \cline{2-3}
        & \multirow{1}{*}{-V} & \multirow{1}{*}{
            \begin{tabular}{c}
                Display version information.
            \end{tabular}
        } \\
        \cline{2-3}
        & \multirow{1}{*}{-v} & \multirow{1}{*}{
            \begin{tabular}{c}
                Invoke verbose mode.
            \end{tabular}
        } \\
        \hline
    \end{tabular}
    \caption{List of \gls{lsb} non-compliant features in ubase grouped by program.}
\end{table}

% SBASE

\begin{table}[!ht]
    \centering
    \begin{tabular}{|c|c|c|}
        \hline
        Program & Option/s & Description or discrepancy \\
        \hline
        \hline
        %%%%%%%%%%%%%%%%%%%%
        \multirow{2}{*}{cp}
        %%%%%%%%%%%%%%%%%%%%
        & \multirow{2}{*}{-i} & \multirow{2}{*}{
            \begin{tabular}{c}
                Prompt when copying to any existing \\
                destination file.
            \end{tabular}
        } \\
        & & \\
        \hline
        \hline
        %%%%%%%%%%%%%%%%%%%%%%%
        \multirow{2}{*}{ed}
        %%%%%%%%%%%%%%%%%%%%%%%
        & \multirow{1}{*}{g, v} & \multirow{1}{*}{
            \begin{tabular}{c}
                Operate on single commands.
            \end{tabular}
        } \\
        \cline{2-3}
        & \multirow{1}{*}{E, e, r, W, w} & \multirow{1}{*}{
            \begin{tabular}{c}
                Cannot accept shell escapes.
            \end{tabular}
        } \\
        \hline
        \hline
        %%%%%%%%%%%%%%%%%%%%
        \multirow{3}{*}{getconf}
        %%%%%%%%%%%%%%%%%%%%
        & \multirow{3}{*}{-v} & \multirow{3}{*}{
            \begin{tabular}{c}
                Indicate a specific specification and version for \\
                which configuration variables are to be \\
                determined.
            \end{tabular}
        } \\
        & & \\
        & & \\
        \hline
        \hline
        \multicolumn{3}{|c|}{\dots} \\
        \hline
    \end{tabular}
\end{table}

\begin{table}[!ht]
    \centering
    \begin{tabular}{|c|c|c|}
        \hline
        Program & Option/s & Description or discrepancy \\
        \hline
        \hline
        %%%%%%%%%%%%%%%%%%%%%%%
        \multirow{13}{*}{ls}
        %%%%%%%%%%%%%%%%%%%%%%%
        & \multirow{3}{*}{-C} & \multirow{3}{*}{
            \begin{tabular}{c}
                Write multi-text-column output with entries \\
                sorted down the columns, according to the \\
                collating sequence.
            \end{tabular}
        } \\
        & & \\
        & & \\
        \cline{2-3}
        & \multirow{2}{*}{-m} & \multirow{2}{*}{
            \begin{tabular}{c}
                Stream output format; list files across \\
                the page, separated by commas.
            \end{tabular}
        } \\
        & & \\
        \cline{2-3}
        & \multirow{2}{*}{-p} & \multirow{2}{*}{
            \begin{tabular}{c}
                Write a slash (/) after each filename \\
                if that file is a directory.
            \end{tabular}
        } \\
        & & \\
        \cline{2-3}
        & \multirow{2}{*}{-s} & \multirow{2}{*}{
            \begin{tabular}{c}
                Indicate the total number of file system \\
                blocks consumed by each file displayed.
            \end{tabular}
        } \\
        & & \\
        \cline{2-3}
        & \multirow{4}{*}{-x} & \multirow{4}{*}{
            \begin{tabular}{c}
                The same as -C, except that the \\
                multi-text-column output is produced \\
                with entries sorted across, \\
                rather than down, the columns.
            \end{tabular}
        } \\
        & & \\
        & & \\
        & & \\
        \hline
        \hline
        %%%%%%%%%%%%%%%%%%%%%%%
        \multirow{10}{*}{install}
        %%%%%%%%%%%%%%%%%%%%%%%
        & \multirow{1}{*}{-b} & \multirow{1}{*}{
            \begin{tabular}{c}
                Equivalent to -{}-backup=existing.
            \end{tabular}
        } \\
        \cline{2-3}
        & \multirow{3}{*}{-p} & \multirow{3}{*}{
            \begin{tabular}{c}
                Copy the access and modification \\
                times of SOURCE files to \\
                corresponding destination files.
            \end{tabular}
        } \\
        & & \\
        & & \\
        \cline{2-3}
        & \multirow{2}{*}{-S} & \multirow{2}{*}{
            \begin{tabular}{c}
                Equivalent to -{}-backup=existing, except \\
                if a simple suffix is required, use SUFFIX.
            \end{tabular}
        } \\
        & & \\
        \cline{2-3}
        & \multirow{2}{*}{-s} & \multirow{2}{*}{
            \begin{tabular}{c}
                Strips symbol tables, only for 1st and \\
                2nd formats.
            \end{tabular}
        } \\
        & & \\
        \cline{2-3}
        & \multirow{2}{*}{-v} & \multirow{2}{*}{
            \begin{tabular}{c}
                Print the name of each file before \\
                copying it to \glsxtrshort{stdout}.
            \end{tabular}
        } \\
        & & \\
        \hline
        \hline
        %%%%%%%%%%%%%%%%%%%%
        \multirow{2}{*}{mv}
        %%%%%%%%%%%%%%%%%%%%
        & \multirow{2}{*}{-i} & \multirow{2}{*}{
            \begin{tabular}{c}
                Prompt for confirmation if the \\
                destination path exists.
            \end{tabular}
        } \\
        & & \\
        \hline
        \hline
        %%%%%%%%%%%%%%%%%%%%
        \multirow{2}{*}{od}
        %%%%%%%%%%%%%%%%%%%%
        & \multirow{1}{*}{-v} & \multirow{1}{*}{
            \begin{tabular}{c}
                Always enabled.
            \end{tabular}
        } \\
        \cline{2-3}
        & \multirow{1}{*}{-t} & \multirow{1}{*}{
            \begin{tabular}{c}
                'd' interpreted as 'u'.
            \end{tabular}
        } \\
        \hline
        \hline
        %%%%%%%%%%%%%%%%%%%%
        \multirow{1}{*}{rm}
        %%%%%%%%%%%%%%%%%%%%
        & \multirow{1}{*}{-i} & \multirow{1}{*}{
            \begin{tabular}{c}
                Prompt before every removal.
            \end{tabular}
        } \\
        \hline
        \hline
        %%%%%%%%%%%%%%%%%%%%
        \multirow{12}{*}{tar}
        %%%%%%%%%%%%%%%%%%%%
        & \multirow{2}{*}{c, f, h, m, t, x, z} & \multirow{2}{*}{
            \begin{tabular}{c}
                Single dash prefix required for \\
                option selection.
            \end{tabular}
        } \\
        & & \\
        \cline{2-3}
        & \multirow{2}{*}{b} & \multirow{2}{*}{
            \begin{tabular}{c}
                Use the first file operand as \\
                the blocking factor for tape records.
            \end{tabular}
        } \\
        & & \\
        \cline{2-3}
        & \multirow{1}{*}{l} & \multirow{1}{*}{
            \begin{tabular}{c}
                Report if not all links can be resolved.
            \end{tabular}
        } \\
        \cline{2-3}
        & \multirow{2}{*}{o} & \multirow{2}{*}{
            \begin{tabular}{c}
                Assign extracted files to the \\
                user running the program.
            \end{tabular}
        } \\
        & & \\
        \cline{2-3}
        & \multirow{1}{*}{r} & \multirow{1}{*}{
            \begin{tabular}{c}
                Append files to the end of an archive.
            \end{tabular}
        } \\
        \cline{2-3}
        & \multirow{2}{*}{u} & \multirow{2}{*}{
            \begin{tabular}{c}
                Add the named file or files to the \\
                archive if they are not already there.
            \end{tabular}
        } \\
        & & \\
        \cline{2-3}
        & \multirow{1}{*}{v} & \multirow{1}{*}{
            \begin{tabular}{c}
                Invoke verbose mode.
            \end{tabular}
        } \\
        \cline{2-3}
        & \multirow{1}{*}{w} & \multirow{1}{*}{
            \begin{tabular}{c}
                Invoke interactive mode.
            \end{tabular}
        } \\
        \hline
        \hline
        %%%%%%%%%%%%%%%%%%%%
        \multirow{2}{*}{xargs}
        %%%%%%%%%%%%%%%%%%%%
        & \multirow{2}{*}{-p} & \multirow{2}{*}{
            \begin{tabular}{c}
                Asked whether to execute utility \\
                at each invocation.
            \end{tabular}
        } \\
        & & \\
        \hline
    \end{tabular}
    \caption{List of \gls{lsb} non-compliant features in sbase grouped by program.}
\end{table}
