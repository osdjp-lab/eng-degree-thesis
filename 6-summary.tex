\section{Summary}\label{Summary}

The purpose of this project was to create a system for Linux learners consisting of a minimal set of software components reduced down to their indispensable functionality. For that reason the \gls{lsb} Core module of the \gls{lsb} specification was selected as the starting point for the system's design. Next standard packages required for \gls{lsb} Core compliance were gathered. The resulting set of software was then reduced. Where available, alternatives to the remaining packages were found. After that all of the retained standard and alternative packages were evaluated. Some packages containing potentially suitable software were excluded due to limited modularity (most collection entries in table \ref{table:Alternative packages}) while for others no replacements could be found (most entries in table \ref{table:Common packages}). In effect a number of standard and alternative packages were selected and assembled into the final system. The choice of the \glsxtrfull{pil} and the number of \glsxtrfull{loc} as criteria for package selection enabled the creation of a system relying on significantly less code than one built exclusively with standard packages. Optimistically the project at present can be considered the first step in the creation of a system and set of software geared towards Linux learners. As such the initially stated goal was not reached in its entirety. However good progress was made on the path towards reaching it.
