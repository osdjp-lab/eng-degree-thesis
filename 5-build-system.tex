\section{Build system}\label{Build system}

An overview of the components of the \gls{lfl} build system as well as of the methodology of system creation.

\subsection{Layout}

The build system can be subdivided into static and dynamic components.

\subsubsection{Static}

The first are part of the repository prior to running the system build:

\begin{itemize}
    \item \textbf{README.md} - general project information.
    \item \textbf{LICENSE} - project use and distribution terms.
    \item \textbf{Makefile} - listing of make targets with associated scripts.
    \item \textbf{build-scripts} - scripts responsible for each step in the creation of the system.
    \item \textbf{src-lists} - lists of software source code locations.
    \item \textbf{make-scripts} - makefile target scripts.
    \item \textbf{pkg-scripts} - individual package build and install scripts.
    \item \textbf{config-files} - system configuration files.
    \item \textbf{chroot-scripts} - chroot system initialisation scripts.
\end{itemize}

\subsubsection{Dynamic}

The second are created during the system build:

\begin{itemize}
    \item \textbf{archive} - compressed tarballs of source code packages.
    \item \textbf{sources} - uncompressed source code packages or git repositories.
    \item \textbf{rootfs} - the root filesystem of the \gls{lfl} system.
    \item \textbf{isoimage} - all files necessary to generate the \gls{iso} disk image.
    \item \textbf{linux-for-learners.iso} - the \gls{lfl} disk image.
\end{itemize}

\subsection{Procedure}

A step by step description of the build process.

\subsubsection{Package acquisition}

In the first step the packages listed in the wget-sources and git-sources files under the src-lists directory are downloaded. The packages available as compressed tarballs, listed in wget-sources, are downloaded by wget into the newly created archive directory. After that the sources directory is created and the git \glspl{url} of packages only available as git repositories are sequentially parsed for their repository names, checked whether they exist in the sources directory and if missing cloned.

% All software packages required for the system are downloaded. Where available the packages are obtained as compressed tarballs and stored into the archive directory otherwise they are cloned from online repositories via git into the sources directory.

\subsubsection{Source code extraction}

In the next step the source code packages contained in compressed tar archives located in the archive directory are extracted into the sources directory. In this step the same as in the previous the extraction is carried out in a loop. At first the stripped down name of the archive omitting the version number of the package, if present, is extracted and made into a \gls{regex}. Next the generated \gls{regex} is used to find the directory corresponding to the archive in the sources directory. If the directory exists the archive is skipped otherwise the archive is extracted.

% All tarballs found in the archive directory are extracted into the sources directory, while in the case of git repositories a selected commit is checked out. After that all source code patches are applied.

\subsubsection{Root filesystem creation}

A directory tree based upon the \gls{fhs} \cite{fhs} is created in the rootfs directory. This filesystem includes the following optional directories: /home, /lib64, /root, /usr/lib64, /usr/src and excludes the following required directories: /etc/opt, /etc/sgml, /etc/X11, /etc/xml, /srv, /usr/local.

\subsubsection{Software installation}

The scripts stored in the pkg-scripts directory are processed in the order specified by their assigned numbers. The contents of each script is read into a bash variable and a \gls{regex} based upon the name of the script excluding the number is generated. The \gls{regex} is used to change the current working directory to that of the corresponding package. Once there the script stored in the bash variable is executed. This process results in each piece of software being built in its respective subdirectory of sources and being installed alongside it's associated manual pages into the root filesystem.

% Each piece of software is built in its respective subdirectory of sources, after which the compiled static binaries are installed in place in the root filesystem along with their associated manual pages.

\subsubsection{System configuration}

The contents of each subdirectory of config-files is recursively copied over into the corresponding directories of rootfs. The ownership of all files and directories of rootfs is changed to that of the root user.

% All configuration files are copied from the config-files directory tree into corresponding directories in the root filesystem. The ownership of files and directories in rootfs is changed to that of the root user.

\subsubsection{Chroot initialisation}

The chroot-scripts directory is copied over to the root filesystem under the /usr/src path. The /proc and /sys virtual filesystems are mounted and the /dev and /dev/pts directories are bind mounted within the rootfs directory at their corresponding mount points. The script enters chroot, clears the host environment and sets the HOME, TERM and PATH variables. Within the created environment the 0-chroot-init script is run. This script in turn sequentially executes the remainder of the scripts in chroot-scripts. Finally after exiting chroot all filesystems and directories mounted under rootfs are unmounted.

% The final steps required to set up the system are run in isolation on the created root filesystem via chroot and include the indexing of manual pages and the addition of the root user.

\subsubsection{Packaging}

All files and directories in rootfs are found and packaged into a cpio archive which in turn gets compressed via gzip into the rootfs.gz file in the isoimage directory. The contents of the isoimage directory are assembled into the final linux-for-learners.iso file using the xorriso utility.

% All of the required system components located in the isoimage directory are packaged into an \glsxtrshort{iso} 9660 disk image.

%\noindent
Contents of the isoimage directory:

\begin{itemize}
    \item \textbf{isolinux.bin} - bootloader binary
    \item \textbf{ldlinux.c32} - bootloader dynamic library
    \item \textbf{isolinux.cfg} - bootloader configuration file
    \item \textbf{kernel.gz} - self-extracting compressed Linux kernel image
    \item \textbf{rootfs.gz} - compressed root filesystem archive 
\end{itemize}
