\section{Build system}\label{Build system}

\subsection{Layout}

Composition of the Linux for Learners project.

\subsubsection{Static}

Components in the repository prior to system build:

\begin{itemize}
    \item \textbf{README.md} - general project information.
    \item \textbf{LICENSE} - project use and distribution terms.
    \item \textbf{Makefile} - listing of make targets with associated scripts.
    \item \textbf{build-scripts} - scripts responsible for each step in the creation of the system.
    \item \textbf{src-lists} - lists of software source code locations.
    \item \textbf{make-scripts} - makefile target scripts.
    \item \textbf{pkg-scripts} - individual package build and install scripts.
    \item \textbf{config-files} - system configuration files.
\end{itemize}

\subsubsection{Dynamic}

Components created dynamically during system creation:

\begin{itemize}
    \item \textbf{archive} - compressed tarballs of source code packages.
    \item \textbf{sources} - uncompressed source code packages or git repositories.
    \item \textbf{rootfs} - the root filesystem of the Linux for Learners system.
    \item \textbf{isoimage} - all files necessary to generate the iso disk image.
    \item \textbf{linux-for-learners.iso} - the Linux for Learners disk image.
\end{itemize}

\subsection{Procedure}

A step by step description of the build process.

\subsubsection{Package acquisition}

All software packages required for the system are downloaded. Where available the packages are gathered in the form of compressed tarballs into the archive directory otherwise they are git cloned from online repositories into the sources directory.

\subsubsection{Source code extraction}

All tarballs found in the archive directory are extracted into the sources directory, while in the case of git repositories a selected commit is checked out. After that all source code patches are applied.

\subsubsection{Root filesystem creation}

The base FHS filesystem layout is created in the rootfs directory.

\subsubsection{Software installation}

Each piece of software is build in its respective subdirectory of sources, after which the compiled static binaries are installed into place in the root filesystem along with their associated man-pages.

\subsubsection{Configuration}

All configuration files are copied from the config-files directory tree into corresponding directories in the root filesystem.

\subsubsection{Packaging}

All of the required system components located in the isoimage directory are packaged into an ISO 9660 disk image.

\noindent
Contents of the isoimage directory:

\begin{itemize}
    \item \textbf{isolinux.bin} - bootloader binary
    \item \textbf{ldlinux.c32} - bootloader dynamic library
    \item \textbf{isolinux.cfg} - bootloader configuration file
    \item \textbf{kernel.gz} - self-extracting compressed Linux kernel image
    \item \textbf{rootfs.gz} - compressed root filesystem archive 
\end{itemize}
