\section{Introduction}\label{Introduction}

Even though many Linux distributions are currently available few are intended for use as a teaching resource. Of those that are, most focus on goals other than software implementation. Projects like Busybox and Toybox, which contain minimal reimplementations of commonly used software primarily intended for embedded use, focus on security, portability as well as adherence to standards including POSIX-2008, LSB4.1 and SUSv3. Others like the Linux From Scratch (LFS) family of operating systems focus on the methodology of system creation using standard software implementations. As a result none of them in their initial state are suitable for the purposes of this project.

\subsection{Free and open-source software (FOSS)}

FOSS is software that simultaneously adheres to two definitions: the definition of free software provided by the Free Software Foundation (FSF) and the definition of open-source software provided by the Open Source Initiative (OSI).

Both definition overlap in many terms of which the following are most significant:

\begin{itemize}
    \item Free to use for any purpose.
    \item Free access to source code.
    \item Freedom to redistribute.
    \item Freedom to distribute derivative works.
\end{itemize}

FOSS is distributed under one or many compatible licenses. The following are a few key examples of FOSS licences:

\begin{itemize}
    \item MIT
    \item GPL
    \item BSD
    \item Apache
\end{itemize}

All of the components used in the creation of this project are available under free and open-source software licensing terms.

\subsection{Operating system}

* Components of an operating system

** Bootloader

** Kernel:

There are many open source operating systems available online Linux, FreeBSD, MINIX, RedoxOS and FreeDOS to name a few 

** Software (init system, logger, display server, utilities, user programs)

\subsection{Linux}

* Why Linux?


* Types of distributions (general purpose, hosting (server), business oriented, security oriented, education, etc.)

Currently there are many Linux distributions available online most of which are built for general use, business or security. Few however are built for the purpose of educating about the implementation and creation of Linux systems. 

* Source based distributions (LFS, Gentoo, NixOS, CRUX, GoboLinux, GNU Guix, Source Mage, Calculate Linux)

* Why my system is different from the above (different target user base, significanly smaller and easier to setup, ease of code modification, chroot and iso options, etc.)


\subsection{Software}

* Why open source software?

* Characteristics of commonly used software (source code, build systems, features, etc.) for example primary GNU software packages (glibc, gcc, binutils, bash, coreutils, diffutils, findutils, gawk, etc.)

A set of common goals underly the development of nearly all software:

\begin{itemize}
    \item Maintenence (patching of bugs).
    \item Optimization (performance).
    \item Extension (introduction of features for versatility or compatibilty with alternative implementations; porting of software to alternative operating systems, etc.).
\end{itemize}

As a result over the course of many years of development the code base of any project increases in size and complexity.

* Graph of selected GNU packages size evolution since inception.

* Changes to mainstream software packages resulting in an abundance of excess code (maintenence, optimization, extension etc.)

* Intended goal (singular build target (Linux) resulting in a minimal build system, reduced complexity, reduced feature set with a goal of adhering to the Unix philosophy of minimalizm and modularity)


\subsection{Code metrics}

Criteria of importance for learning linux software implementation details.

\subsubsection{Software comprehension}

\subsubsection{Understandability}

\subsubsection{Readability}

\subsubsection{Cognitive complexity}

* Weyuker's nine desirable properties of complexity metrics.

* Halstead complexity measures.

* McCabe cyclomatic complexity.

* Cognitive weight and functional size (A new measure of software complexity based on cognitive weights).
