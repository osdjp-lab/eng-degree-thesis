\section{Introduction}\label{Introduction}

\subsection{Free and open-source software (FOSS)}

FOSS is software that simultaneously adheres to two definitions: the definition of free software provided by the Free Software Foundation (FSF) \cite{fsf} and the definition of open-source software provided by the Open Source Initiative (OSI) \cite{osi}.

Both definitions overlap in many terms of which the following are most significant:

\begin{itemize}
    \item Free access to source code.
    \item Freedom to redistribute.
    \item Freedom to use for any purpose.
    \item Freedom to distribute derivative works.
\end{itemize}

FOSS is distributed under one or many compatible licences. The following are a few key examples of FOSS licences:

\begin{itemize}
    \item CC0, Unlicense - public domain or equivalent.
    \item BSD, MIT, Apache - permissive.
    \item AGPL, GPL - protective (copyleft).
\end{itemize}

All of the components used in the creation of this project are available under free and open-source software licensing terms.

\subsection{Standard software packages}

The FOSS communities are the source of a great number and variety of actively maintained software implementations. Among these only a small subset is considered essential and is included by default in the majority of modern day open-source operating systems. This selection of software for Unix and Linux systems is defined in the following standards:

\begin{itemize}
    \item Single UNIX Specification (SUS) \cite{susv4}.
    \item Linux Standard Base (LSB) specification \cite{lsb}.
\end{itemize}

These specifications extend far beyond the scope of this project as such only the LSB Core subset of the LSB standard has been chosen for reference in the creation of the Linux for learners system.

\begin{table}[H]
    \centering
    %\begin{center}
        \begin{tabular}{|c|c|c|c|}
            \hline
            at & bash & bc & binutils \\
            \hline
            coreutils & cpio & diffutils & ed \\
            \hline
            fcron & file & findutils & gawk \\
            \hline
            grep & gzip & lsb-tools & m4 \\
            \hline
            man-db & ncurses & nspr & nss \\
            \hline
            pam & pax & procps & psmisc \\
            \hline
            sed & sendmail & shadow & tar \\
            \hline
            time & util-linux & zlib \\
            \cline{1-3}
        \end{tabular}
        \caption{A selection of commonly used FOSS packages required for LSB Core compliance.}
    %\end{center}
\end{table}

Over half of the packages needed to fulfil the requirements of LSB Core originate from the GNU project, with the remainder being the work of other organizations or groups. Many have been in development for over two decades and in that time have undergone extensive growth as the result of feature additions, bug fixes or general maintenance. In effect their source code has increased significantly in both size and complexity making it for the most part unsuitable for use as a teaching resource.

\subsection{Software bloat}

Software bloat can be defined as a collection of factors impacting the functionality, performance, development and/or maintenance of a piece of software. Much effort has been put forward in order to prevent and/or alleviate its effects. This has resulted in the formulation of numerous guidelines (\cite[C++ Core Guidelines]{cpp-guidelines}, \cite[Haskell Guidelines]{haskel-guidelines}), principles (\cite["Keep it Simple, Stupid" (KISS)]{kiss}, \cite[The Unix philosophy]{unix-philosophy}, \cite["Worse is better"]{worse-is-better}, \cite["Don't Repeat Yourself" (DRY)]{dry}, \cite["You aren't gonna need it" (YAGNI)]{yagni}, \cite[SOLID]{solid}) and paradigms (\cite[Imperative programming]{imperative}, \cite[Declarative programming]{declarative}) as well as the publication of a multitude of books \cite{clean-code,clean-architecture,pragmatic-programmer,extreme-programming,unix-programming} and articles \cite{Pike2007ProgramDI,Milicchio2007TheUK,Thayer2004FourUP,McGrenere2000AreWA,Quach2018DebloatingST,McGrenere2000BloatTO,Quach2019BloatFA}. In effect many different kinds of software bloat have been defined. Some of the most commonly studied include inefficient usage of resources, bad software design as well as the inclusion of excessive or unwanted features. Since software bloat in large part depends on the point of view of the observer other types can also be delimited. The following are a subset of factors contributing to the applicability of a given piece of software for use as a teaching resource:

\begin{itemize}
    \item Insufficient documentation.
    \item Excess features.
    \item Unclear optimization.
    \item Unneeded portability considerations.
    \item Excess fall-back implementations.
    \item Use of multiple programming languages.
    \item Bad formatting or naming conventions.
    \item Complex build systems.
    \item Convoluted code structuring.
\end{itemize}

Each of these points require carefully defined means of quantification.

\subsection{Source code evaluation}

Numerous metrics how been proposed for the assessment of source code.
The following are a few most applicable to the goals of this project.

\begin{itemize}
    \item Lines of code (LOC)
    \item McCabe cyclomatic complexity
    \item Halstead complexity measures
    \item Maintainability index
    \item Function points
    \item Cognitive functional size (CFS)
\end{itemize}

\subsection{Operating systems}

Operating systems are a complex composition of different programs (software) which enable the utilization of a computer (hardware). They consist of the following elements:

\begin{itemize}
    \item Bootloader - responsible for loading the kernel.
    \item Kernel - the foundation of the operating system, responsible for managing access to available resources as well as providing a clear and consistent abstraction of the hardware for the creation of software.
    \item User space - software designed for the user or users of the operating system.
\end{itemize}

Currently many open source operating systems are available online for e.g. Linux, FreeBSD, MINIX, RedoxOS and FreeDOS. Of these the most commonly used and contributed to is Linux.

\subsection{Linux}

Colloquially the term Linux is used to refer to systems utilizing the Linux kernel. These differ in terms of their implementation. Whether that be the result of supporting different platforms, utilizing different bootloaders, kernel configurations or software packages.

Many Linux distributions are currently available. Few however have goals which overlap with the intended purpose of this project, teaching about the implementation of software and the creation of a Linux system. Those that do, fall into two categories:

\begin{itemize}
    \item Source based
    \item Embedded
\end{itemize}

The first are characterized by modular software installation from source code. They include LFS, Gentoo, NixOS, CRUX and GNU Guix to name a few. Of these the Linux From Scratch (LFS) family of operating systems, are of greatest interest due to their focus on a step-by-step approach to system creation using standard software implementations. 

The second group include systems like Busybox and Toybox which are primarily intended for embedded use and contain minimal reimplementations of commonly used software.

This project stands apart from those mentioned above. Unlike LFS it intends to incorporate only minimal software solutions and unlike Busybox and Toybox aims to reduce their implementations to a theoretical minimal cognitive complexity not bound by goals like standards compliance, efficiency or security. As a result none of the preceding systems are suitable for the purposes of this project.
