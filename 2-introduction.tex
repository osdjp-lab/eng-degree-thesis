\section{Introduction}\label{Introduction}

The conceptual foundation for this system stems from the Linux From Scratch (LFS) project \cite{lfs}, it's derivative project Automated Linux From Scratch (ALFS) \cite{alfs} as well as a seperate minimal system creation project called Minimal Linux Live (MLL) \cite{mll}. Similarly to the aformentioned projects as well as many other Linux systems this project relies on the following standards for some of its implementation details:

\begin{itemize}
    \item Linux Standard Base (LSB) \cite{lsb} - for its selection of included packages and features.
    \item Filesystem Hierarchy Standard (FHS) \cite{fhs} - for its directory layout.
    \item POSIX
    \item Single Unix Specification (SUS)
\end{itemize}


A set of common goals underly the development of nearly all software:

\begin{itemize}
    \item Performance optimization.
    \item Introduction of new features.
\end{itemize}

As a result over many years of development the code base of any project is going to increase in size and possibly complexity.

An example of this are the GNU software packages.

Overview of primary GNU packages (glibc, gcc, binutils, bash, coreutils, diffutils, findutils, gawk)

Changes to mainstream software packages since their inception resulting in an abundance of excess code (depricated features etc.)

Enormous code base causing labour intensive maintenance as well as addition of novel feautures, removal of bugs.


