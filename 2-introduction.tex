\section{Introduction}\label{Introduction}

\subsection{Free and open-source software (FOSS)}

FOSS is software that simultaneously adheres to two definitions: the definition of free software provided by the Free Software Foundation (FSF) \cite{fsf} and the definition of open-source software provided by the Open Source Initiative (OSI) \cite{osi}.

Both definitions overlap in many terms of which the following are most significant:

\begin{itemize}
    \item Free access to source code.
    \item Freedom to redistribute.
    \item Freedom to use for any purpose.
    \item Freedom to distribute derivative works.
\end{itemize}

FOSS is distributed under one or many compatible licences. The following are a few key examples of FOSS licences:

\begin{itemize}
    \item CC0, Unlicense - public domain or equivalent.
    \item BSD, MIT, Apache - permissive.
    \item AGPL, GPL - protective (copyleft).
\end{itemize}

All of the components used in the creation of this project are available under free and open-source software licensing terms.

\subsection{Standard software packages}

The FOSS communities are the source of a great number and variety of actively maintained software implementations. However only a small subset of these is included by default in modern day open-source operating systems. This selection tends to differ from system to system. As such the following subsets of standardized system interfaces have been created in order to provide a uniform foundation for software development and use on Unix and Linux systems:

\begin{itemize}
    \item Single UNIX Specification (SUS) \cite{susv4}.
    \item Linux Standard Base (LSB) specification \cite{lsb}.
\end{itemize}

These specifications extend far beyond the scope of this project as such only the LSB Core subset of the LSB standard has been considered for use in the creation of the Linux for learners system.

\begin{table}[H]
    \centering
    %\begin{center}
        \begin{tabular}{|c|c|c|c|}
            \hline
            at & bash & bc & binutils \\
            \hline
            coreutils & cpio & diffutils & ed \\
            \hline
            fcron & file & findutils & gawk \\
            \hline
            grep & gzip & lsb-tools & m4 \\
            \hline
            man-db & ncurses & nspr & nss \\
            \hline
            pam & pax & procps & psmisc \\
            \hline
            sed & sendmail & shadow & tar \\
            \hline
            time & util-linux & zlib \\
            \cline{1-3}
        \end{tabular}
        \caption{A selection of commonly used FOSS packages required for LSB Core compliance.}
    %\end{center}
\end{table}

Over half of the standard packages needed to fulfil the requirements of LSB Core originate from the GNU project, with the remainder being the work of other organizations or groups. Many have been in development for over two decades and in that time have undergone extensive growth as the result of feature addition, bug fixes or general maintenance. In effect their source code has increased significantly in both size and complexity \cite{Quach2018DebloatingST,Quach2019BloatFA}.

\subsection{Software bloat}

Software bloat can be defined as a collection of factors impacting the functionality, performance, development and/or maintenance of a piece of software \cite{McGrenere2000AreWA,McGrenere2000BloatTO,Quach2018DebloatingST,Quach2019BloatFA}. A significant amount of research has been conducted that directly or indirectly evaluates the causes \cite{Mitchell2010FourTL} and effects \cite{Quach2019BloatFA} as well as means of prevention \cite{Pike2007ProgramDI,Milicchio2007TheUK} and alleviation \cite{Quach2018DebloatingST} of software bloat. As a result many different kinds of software bloat have been defined. Some of the most commonly studied include inefficient usage of resources, bad software design as well as feature creep. Since software bloat in large part depends on the point of view of the observer other types can also be delimited. The following are a subset of factors contributing to the applicability of a given piece of software for use as a teaching resource:

\begin{itemize}
    \item Insufficient documentation.
    \item Excess features.
    \item Unclear optimization.
    \item Extraneous portability considerations.
    \item Excess fall-back implementations.
    \item Use of multiple programming languages.
    \item Convoluted code structuring (unsuitable formatting and/or naming conventions).
    \item Complex build systems.
\end{itemize}

\subsection{Source code evaluation}

In order to evaluate a given piece of source code its qualitative and quantitative properties need to be assessed or measured. 

\subsubsection{Qualities}

Consist of non-numeric values used to describe the source code. Below are a few of the most relevant:

\begin{itemize}
    \item Programming language.
    \item Coding convention.
    \item Comments and documentation.
    \item Software architecture.
\end{itemize}

\subsubsection{Quantities}

Consist of discrete measurable values which by themself or in composition can form distinct metrics. The following are a few most applicable to the goals of this project.

\begin{itemize}
    \item Lines of code (LOC)
    \item McCabe cyclomatic complexity
    \item Halstead complexity measures
    \item Maintainability index
    \item Function points
    \item Cognitive functional size (CFS)
\end{itemize}

\subsubsection{Automation}

The following is a selection of free and open-source programs for the automated evaluation of source code properties:

\begin{itemize}
    \item tokei \cite{tokei}
    \item cloc \cite{cloc}
    \item ohcount \cite{ohcount}
    \item cccc \cite{cccc}
\end{itemize}

\subsection{Operating systems}

Operating systems are a complex composition of different programs (software) which enable the utilization of a computer (hardware). They consist of the following elements:

\begin{itemize}
    \item Bootloader - responsible for loading the kernel.
    \item Kernel - the foundation of the operating system, responsible for managing access to available resources as well as providing a clear and consistent abstraction of the hardware for the creation of software.
    \item User space - software designed for the user or users of the operating system.
\end{itemize}

Currently many open source operating systems are available online for e.g. Linux, FreeBSD, MINIX, RedoxOS and FreeDOS. Of these the most commonly used and contributed to is Linux.

\subsection{Linux}

Colloquially the term Linux is used to refer to systems utilizing the Linux kernel. These differ in terms of their implementation. Whether that be the result of supporting different platforms, utilizing different bootloaders, kernel configurations or software packages.

Many Linux distributions are currently available. Few however have goals which overlap with the intended purpose of this project, teaching about the implementation of software and the creation of a Linux system. Those that do, fall into two categories:

\begin{itemize}
    \item Source based
    \item Embedded
\end{itemize}

The first are characterized by modular software installation from source code. They include LFS, Gentoo, NixOS, CRUX and GNU Guix to name a few. Of these the Linux From Scratch (LFS) family of operating systems, are of greatest interest due to their focus on a step-by-step approach to system creation using standard software implementations. 

The second group include systems like Busybox and Toybox which are primarily intended for embedded use and contain minimal reimplementations of commonly used software.

This project stands apart from those mentioned above. Unlike LFS it intends to incorporate only minimal software solutions and unlike Busybox and Toybox aims to reduce their implementations to a theoretical minimal cognitive complexity not bound by goals like standards compliance, efficiency or security. As a result none of the preceding systems are suitable for the purposes of this project.
