\section{Introduction}\label{Introduction}

\subsection{Free and open-source software (FOSS)}

All of the components used in the creation of this project are available under free and open-source software licensing terms.

FOSS is software that simultaneously adheres to two definitions: the definition of free software provided by the Free Software Foundation (FSF) and the definition of open-source software provided by the Open Source Initiative (OSI).

Both definition overlap in many terms of which the following are most significant:

\begin{itemize}
    \item Free access to source code.
    \item Freedom to redistribute.
    \item Freedom to use for any purpose.
    \item Freedom to distribute derivative works.
\end{itemize}

FOSS is distributed under one or many compatible licences. The following are a few key examples of FOSS licences:

\begin{itemize}
    \item CC0, Unlicense - public domain or equivalent.
    \item BSD, MIT, Apache - permissive
    \item AGPL, GPL - protective (copyleft)
\end{itemize}

* Advantages and disadvantages of FOSS

\subsection{Software development}

* Increase in size and complexity of any project undergoing continious development.

* Changes to mainstream software packages resulting in an abundance of excess code (maintenence, optimization, extension etc.)

* Intended goal (singular build target (Linux) resulting in a minimal build system, reduced complexity, reduced feature set with a goal of adhering to the Unix philosophy of minimalizm and modularity)

\subsection{Standard FOSS software packages}

* Characteristics of commonly used software (source code, build systems, features, etc.) for example primary GNU software packages (glibc, gcc, binutils, bash, coreutils, diffutils, findutils, gawk, etc.)

* Graph of selected GNU packages size evolution since inception.

\subsection{Source code evaluation}

Numerous metrics how been proposed for the assessment of source code.
The following are a few most applicable to the goals of this project.

\begin{itemize}
    \item Lines of code (LOC)
    \item McCabe cyclomatic complexity
    \item Halstead complexity measures
    \item Maintainability index
    \item Function points
    \item Cognitive functional size (CFS)
\end{itemize}

\subsection{Operating systems}

Operating systems are a complex composition of different programs (software) which enable the utilization of a computer (hardware). They consist of the following elements:

\begin{itemize}
    \item Bootloader
    \item Kernel
    \item User space
\end{itemize}

There are many open source operating systems available online including Linux, FreeBSD, MINIX, RedoxOS and FreeDOS.

\subsection{Linux systems}

Many Linux distributions are currently available. Few however have goals which overlap with the intended purpose of this project, teaching about the implementation of software and the creation of a Linux system. Those that do, fall into two categories:

\begin{itemize}
    \item Source based
    \item Embedded
\end{itemize}

The first are characterized by modular software installation from source code. They include LFS, Gentoo, NixOS, CRUX and GNU Guix to name a few. Of these the Linux From Scratch (LFS) family of operating systems, are of greatest interest due to their focus on a step-by-step approach to system creation using standard software implementations. 

The second group include systems like Busybox and Toybox which are primarily intended for embedded use and contain minimal reimplementations of commonly used software.

This project stands apart from those mentioned above. Unlike LFS it intends to incorporate only minimal software solutions and unlike Busybox and Toybox aims to reduce their implementations to a theoretical minimal cognitive complexity not bound by goals like standards compliance, efficiency or security. As a result none of the preceeding systems are suitable for the purposes of this project.
