\section{Introduction}\label{Introduction}

The focus of this thesis is the creation of a Linux-based system called \enquote{\glsxtrlong{lfl}}\cite{lfl}. This system consists of minimal implementations of commonly used software available under free and open source software licensing terms and is intended to serve as a teaching resource for study and modification by junior programmers.

\subsection{Free and open source software}

\gls{foss} is software that simultaneously adheres to two definitions: the definition of free software \cite{fsf-free-software} provided by the \gls{fsf} \cite{fsf} and the definition of open source software \cite{osi-open-source} provided by the \gls{osi} \cite{osi}. Both definitions overlap in many terms of which the following are most significant:

\begin{itemize}
    \item Free access to source code.
    \item Freedom to redistribute.
    \item Freedom to use for any purpose.
    \item Freedom to distribute derivative works.
\end{itemize}

\gls{foss} is distributed under one or many compatible licences. The following are a few key examples of \gls{foss} licences \cite{spdx-licenses,fsf-licenses}:

\begin{itemize}
    \item Unlicense \cite{unlicense} - public domain \cite{public-domain} equivalent.
    \item \glsxtrshort{mit-licence} \cite{mit}, \glsxtrshort{bsd-2} \cite{bsd-2}, \glsxtrshort{bsd-3} \cite{bsd-3}, Apache \cite{apache} - permissive \cite{osi-permissive}.
    \item \glsxtrshort{mpl} \cite{mpl}, \glsxtrshort{lgpl} \cite{lgpl}, \glsxtrshort{gpl} \cite{gpl}, \glsxtrshort{agpl} \cite{agpl} - protective (copyleft) \cite{gnu-copyleft,eupl-copyleft}.
\end{itemize}

\subsection{Standard software packages}

The \glsxtrshort{foss} communities are the source of a great number and variety of actively maintained software implementations. However only a small subset of these is included by default in modern day open source operating systems. This selection tends to differ from system to system. As such the following subsets of standardised system interfaces have been created in order to provide a uniform foundation, for software development and use, on \glsxtrshort{unix} and Linux systems:

\begin{itemize}
    \item \gls{sus} \cite{susv4}.
    \item \gls{lsb} specification \cite{lsb}.
\end{itemize}

These specifications extend far beyond the scope of this project as such only the \gls{lsb} Core subset of the \gls{lsb} standard has been considered for use in the creation of the \enquote{\glsxtrlong{lfl}} system.

\begin{table}[H]
    \centering
    \begin{tabular}{|c|c|c|c|}
        \hline
        at & bash & bc & binutils \\
        \hline
        coreutils & cpio & diffutils & ed \\
        \hline
        fcron & file & findutils & gawk \\
        \hline
        grep & gzip & lsb-tools & m4 \\
        \hline
        man-db & ncurses & nspr & nss \\
        \hline
        pam & pax & procps & psmisc \\
        \hline
        sed & sendmail & shadow & tar \\
        \hline
        time & util-linux & zlib \\
        \cline{1-3}
    \end{tabular}
    \caption{A selection of commonly used \gls{foss} packages required for \gls{lsb} Core compliance.}
    \label{table:LSB Core}
\end{table}

Over half of the standard packages needed to fulfil the requirements of \gls{lsb} Core originate from the \glsxtrshort{gnu} project, with the remainder being the work of other organisations or groups. Many have been in development for over two decades and in that time have undergone extensive growth as the result of feature addition, bug fixes or general maintenance. In effect their source code has increased significantly in both size and complexity \cite{Succi2001PreliminaryRF,Quach2018DebloatingST,Quach2019BloatFA}.

\subsection{Software bloat}

Software bloat can be defined as a collection of factors impacting the functionality, performance, development and/or maintenance of a piece of software \cite{McGrenere2000AreWA,McGrenere2000BloatTO,Quach2018DebloatingST,Quach2019BloatFA}. A significant amount of research has been conducted that directly or indirectly evaluates the causes \cite{Mitchell2010FourTL} and effects \cite{Quach2019BloatFA} as well as means of prevention \cite{Pike2007ProgramDI,Milicchio2007TheUK} and alleviation \cite{Quach2018DebloatingST} of software bloat. As a result many different kinds of software bloat have been defined. Some of the most commonly studied include inefficient usage of resources, bad software design as well as feature creep. Since software bloat in large part depends on the point of view of the observer other types can also be delimited. The following are a subset of factors contributing to the applicability of a given piece of software for use as a teaching resource:

\begin{itemize}
    \item Insufficient documentation.
    \item Excess features.
    \item Unclear optimisation.
    \item Extraneous portability considerations.
    \item Excess fall-back implementations.
    \item Use of multiple programming languages.
    \item Convoluted code structuring (unsuitable formatting and/or naming conventions).
    \item Complex build systems.
\end{itemize}

\subsection{Source code evaluation}

In order to evaluate a given piece of source code its qualitative and quantitative properties need to be assessed or measured. 

\subsubsection{Qualities}

The first set, qualities consist of non-numeric values used to describe the source code. Below are a few of the most relevant:

\begin{itemize}
    \item Programming language.
    \item Coding convention.
    \item Comments and documentation.
    \item Software architecture.
\end{itemize}

\subsubsection{Quantities}

The second set, quantities consist of discrete measurable values which individually or in composition can form distinct metrics. The following are a few most applicable to the goals of this project.

\begin{itemize}
    \item \Gls{loc} \cite{loc}
    \item \gls{lcm} \cite{lcm-1,lcm-2}
    \item \gls{nof} \cite{nof}
    \item McCabe cyclomatic complexity \cite{mccabe-complexity}
    \item Halstead complexity measures \cite{halstead-complexity}
    \item Maintainability index \cite{maintainability-1,maintainability-2}
    \item Function points \cite{functional-size}
    \item Cognitive functional size \cite{cognitive-complexity}
\end{itemize}

\subsubsection{Automation}

There are many free and open source programs available for the automated evaluation of source code properties. Few however provide options for measuring values outside of the most basic including the number of lines of code or number of lines of comments. Among those that do their scope is usually limited to a single or at most a few languages with similarly few available metrics. Additionally the consistency of values obtained for the same metrics and code base between different revisions of the same tool or between tools providing the same metrics is not guaranteed.

\begin{table}[H]
    \centering
    \begin{tabular}{|c|c|c|}
        \hline
        Program & Supported languages & Supported metrics \\
        \hline
        \hline
        tokei \cite{tokei} & 226 languages
        & \multirow{3}{*}{
            \begin{tabular}{c}
                Lines of code \\
                Lines of comments \\
                Number of files \\
            \end{tabular}
        } \\
        \cline{1-2}
        cloc \cite{cloc} & 249 languages & \\
        \cline{1-2}
        ohcount \cite{ohcount} & 121 languages & \\
        \hline
        \multirow{6}{*}{cccc \cite{cccc}} & \multirow{6}{*}{C, C++, Java}
        & Lines of code \\
        & & Lines of comments \\
        & & Cyclomatic complexity \\
        & & 4/6 Chidamber-Kemerer metrics \\
        & & Fan-in and Fan-out \\
        & & Information flow \\
        \hline
        \multirow{5}{*}{radon \cite{radon}} & \multirow{5}{*}{Python}
        & Lines of code \\
        & & Lines of comments \\
        & & Cyclomatic complexity \\
        & & Halstead metrics \\
        & & Maintainability index \\
        \hline
        \multirow{4}{*}{frama-c \cite{frama-c}} & \multirow{4}{*}{C}
        & Lines of code \\
        & & Cyclomatic complexity \\
        & & Halstead metrics \\
        & & Eva coverage estimate \\
        \hline
    \end{tabular}
    \caption{Programs for automated source code evaluation.}
\end{table}

\subsection{Operating systems}

Operating systems are complex compositions of programs (software) which enable the utilisation of a computer (hardware). They consist of the following elements:

\begin{itemize}
    \item Bootloader - responsible for loading the kernel.
    \item Kernel - the foundation of the operating system, responsible for managing access to available resources as well as providing a clear and consistent abstraction of the hardware for the creation of software.
    \item User space - software designed for the user or users of the operating system.
\end{itemize}

At present many open source operating systems are available for e.g. Linux, FreeBSD, MINIX, RedoxOS and FreeDOS. Of these the most commonly used and contributed to is Linux.

\subsection{Linux}

Colloquially the term Linux is used to refer to systems utilising the Linux kernel. These differ in terms of their implementation. Whether that be the result of supporting different platforms, utilising different bootloaders, kernel configurations or software packages.

Many Linux distributions are currently available. Few however have goals which overlap with the intended purpose of this project, teaching about the implementation of software and the creation of a Linux system. Those that do, fall into two categories:

\begin{itemize}
    \item Source based
    \item Embedded
\end{itemize}

The first are characterised by modular software installation from source code. They include \glsxtrshort{lfs}, Gentoo, NixOS, CRUX and \glsxtrshort{gnu} Guix to name a few. Of these the \glsxtrfull{lfs} family of operating systems, are of greatest interest due to their focus on a step-by-step approach to system creation using standard software implementations. 

The second group includes monolithic systems like Busybox and Toybox which contain minimal reimplementations of commonly used software primarily intended for embedded use.

This project stands apart from those mentioned above. Unlike \glsxtrshort{lfs} it intends to incorporate only minimal software solutions and unlike Busybox and Toybox aims to reduce their implementations to a theoretical minimal cognitive complexity not bound by goals like standards compliance, efficiency or security while retaining modularity. As a result none of the preceding systems entirely fulfil the goals of this project.
