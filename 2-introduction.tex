\section{Introduction}\label{Introduction}

Many Linux distributions are currently available. Few however are built for the purpose of being used as a teaching resource.

Nonetheless there are still numerous systems (which/that), even though not intended, can be used for the purposes of education.

The following groups of systems exhibit commonality with the goals of this project. Providing education about the implementation of software and the creation of Linux systems.

Of these there are two major categories:

\begin{itemize}
    \item Source based distributions
    \item Embedded systems
\end{itemize}

The first set of systems  are the so called source based distributions. They include LFS, Gentoo, NixOS, CRUX and GNU Guix to name a few.  

Of these the Linux From Scratch (LFS) family of operating systems, are of interest due to their focus on a step-by-step approach to guiding and explaining the methodology of system creation using standard software implementations. 

The second set are systems primarily intended for embedded use. They include Busybox, Toybox, Buildroot and Yocto.

Projects like Busybox and Toybox, which contain minimal reimplementations of commonly used software primarily intended for embedded use, focus on security, portability as well as adherence to standards including POSIX-2008, LSB4.1 and SUSv3.

* Why my system is different from the above (different target user base, significanly smaller and easier to setup, ease of code modification, chroot and iso options, etc.)

Is intended to serve as a teaching resource for the purpose of educating about the implementation of software and the creation of Linux systems.

As a result none of them in their initial state are suitable for the purposes of this project.

\subsection{Free and open-source software (FOSS)}

FOSS is software that simultaneously adheres to two definitions: the definition of free software provided by the Free Software Foundation (FSF) and the definition of open-source software provided by the Open Source Initiative (OSI).

Both definition overlap in many terms of which the following are most significant:

\begin{itemize}
    \item Free access to source code.
    \item Freedom to redistribute.
    \item Freedom to use for any purpose.
    \item Freedom to distribute derivative works.
\end{itemize}

FOSS is distributed under one or many compatible licences. The following are a few key examples of FOSS licences:

\begin{itemize}
    \item MIT
    \item GPL
    \item BSD
    \item Apache
\end{itemize}

All of the components used in the creation of this project are available under free and open-source software licensing terms.

\subsection{Operating system}

* Components of an operating system

** Bootloader

** Kernel:

There are many open source operating systems available online Linux, FreeBSD, MINIX, RedoxOS and FreeDOS to name a few 

** Software (init system, logger, display server, utilities, user programs)

\subsection{Software}

* Why open source software?

* Characteristics of commonly used software (source code, build systems, features, etc.) for example primary GNU software packages (glibc, gcc, binutils, bash, coreutils, diffutils, findutils, gawk, etc.)

A set of common goals underly the development of nearly all software:

\begin{itemize}
    \item Maintenence (patching of bugs).
    \item Optimization (performance).
    \item Extension (introduction of features for versatility or compatibilty with alternative implementations; porting of software to alternative operating systems, etc.).
\end{itemize}

As a result over the course of many years of development the code base of any project increases in size and complexity.

* Graph of selected GNU packages size evolution since inception.

* Changes to mainstream software packages resulting in an abundance of excess code (maintenence, optimization, extension etc.)

* Intended goal (singular build target (Linux) resulting in a minimal build system, reduced complexity, reduced feature set with a goal of adhering to the Unix philosophy of minimalizm and modularity)


\subsection{Code metrics}

Criteria of importance for learning linux software implementation details.

\subsubsection{Software comprehension}

\subsubsection{Understandability}

\subsubsection{Readability}

\subsubsection{Cognitive complexity}

* Weyuker's nine desirable properties of complexity metrics.

* Halstead complexity measures.

* McCabe cyclomatic complexity.

* Cognitive weight and functional size (A new measure of software complexity based on cognitive weights).
