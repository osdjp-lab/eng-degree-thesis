\section{Introduction}\label{Introduction}

\subsection{Free and open-source software (FOSS)}

FOSS is software that simultaneously adheres to two definitions: the definition of free software provided by the Free Software Foundation (FSF) and the definition of open-source software provided by the Open Source Initiative (OSI).

Both definition overlap in many terms of which the following are most significant:

\begin{itemize}
    \item Free access to source code.
    \item Freedom to redistribute.
    \item Freedom to use for any purpose.
    \item Freedom to distribute derivative works.
\end{itemize}

FOSS is distributed under one or many compatible licences. The following are a few key examples of FOSS licences:

\begin{itemize}
    \item CC0, Unlicense - public domain or equivalent.
    \item BSD, MIT, Apache - permissive
    \item AGPL, GPL - protective (copyleft)
\end{itemize}

* Advantages and disadvantages of FOSS

All of the components used in the creation of this project are available under free and open-source software licensing terms.

\subsection{Standard software packages}

A large number of commonly used FOSS tools originte from the GNU project. Many of these have been in development for more than two decades growing significanlty over that time. This growth whether it be the result of feature additions, bug fixes, optimizations or general maintenence has made the source code of these tools significantly larger and in effect for the most part unsuitable as a teaching resource.

* Graph of selected GNU packages size evolution since inception.

\subsection{Software bloat}

There are many different kinds of software bloat. A few of the most commonly mentioned include inefficient usage of resources, bad software design as well as the inclusion of excessive or unwanted features. Since software bloat in large part depends on the point of view of the observer other types can also be delimited. In the case of the applicability of a given piece of source code for use as a teaching resource the following factors as well as others should be considered:

\begin{itemize}
    \item Insufficient documentation.
    \item Excess features.
    \item Unclear optimization.
    \item Unneeded portability considerations.
    \item Excess fall-back implementations.
    \item Use of multiple programming lanaguages.
    \item Bad formating or naming conventions.
    \item Complex build systems.
    \item Convoluted code structuring.
\end{itemize}

Each of these points require carefully defined means of quantification.

\subsection{Source code evaluation}

Numerous metrics how been proposed for the assessment of source code.
The following are a few most applicable to the goals of this project.

\begin{itemize}
    \item Lines of code (LOC)
    \item McCabe cyclomatic complexity
    \item Halstead complexity measures
    \item Maintainability index
    \item Function points
    \item Cognitive functional size (CFS)
\end{itemize}

\subsection{Operating systems}

Operating systems are a complex composition of different programs (software) which enable the utilization of a computer (hardware). They consist of the following elements:

\begin{itemize}
    \item Bootloader - responsible for loading the kernel.
    \item Kernel - the foundation of the operating system, responsible for managing access to available resources as well as providing a clear and consistent abstraction of the hardware for the creation of software.
    \item User space - software designed for the user or users of the operating system.
\end{itemize}

Currently many open source operating systems are available online for e.g. Linux, FreeBSD, MINIX, RedoxOS and FreeDOS. Of these the most commonly used and contributed to is Linux.

\subsection{Linux}

Colloquially the term Linux is used to refer to systems utilizing the Linux kernel. These differ in terms of their implementation. Whether that be the result of supporting different platforms, utilizing different bootloaders, kernel configurations or software packages.

Many Linux distributions are currently available. Few however have goals which overlap with the intended purpose of this project, teaching about the implementation of software and the creation of a Linux system. Those that do, fall into two categories:

\begin{itemize}
    \item Source based
    \item Embedded
\end{itemize}

The first are characterized by modular software installation from source code. They include LFS, Gentoo, NixOS, CRUX and GNU Guix to name a few. Of these the Linux From Scratch (LFS) family of operating systems, are of greatest interest due to their focus on a step-by-step approach to system creation using standard software implementations. 

The second group include systems like Busybox and Toybox which are primarily intended for embedded use and contain minimal reimplementations of commonly used software.

This project stands apart from those mentioned above. Unlike LFS it intends to incorporate only minimal software solutions and unlike Busybox and Toybox aims to reduce their implementations to a theoretical minimal cognitive complexity not bound by goals like standards compliance, efficiency or security. As a result none of the preceeding systems are suitable for the purposes of this project.
