\subsection{Packages}\label{Packages}

Initially a small subset of essential standard packages was selected. For each of these packages smaller alternatives were found and evaluated in accordance with the selection critera. The resulting set of software was assembled into the final system.

\subsubsection{Common packages}

Packages retained from selected subset of standard packages:

\begin{table}[!h]
    \centering
    \begin{tabular}{|c||c|}
        \hline
        Package & LOC \\
        \hline
        \hline
        linux \cite{linux} & 22 104 580 \\
        \hline
        sysklogd \cite{sysklogd} & 3 861 \\
        \hline
        bc \cite{bc} & 23 708 \\
        \hline        
        diffutils \cite{diffutils} & 127 219 \\
        \hline
        less \cite{less} & 21 763 \\
        \hline
        patch \cite{patch} & 64 276 \\
        \hline
    \end{tabular}
    \caption{Common packages and their sizes expressed in lines of code (LOC).}
\end{table}

\newpage

\subsubsection{Differing packages}

Packages that differ from the selected subset of standard packages.

\begin{table}[!h]
    \centering
    \begin{tabular}{|c|c||c|c|}
        \hline
        \multicolumn{2}{|c||}{Standard} & \multicolumn{2}{|c|}{Selected} \\
        \hline
        \hline
        Package & LOC & Package & LOC \\
        \hline
        \hline
        grub \cite{grub} & 456 318 & syslinux \cite{syslinux} & 412 273 \\
        \hline
        sysvinit \cite{sysvinit} & 9 052 & sinit \cite{sinit} & 116 \\
        \hline
        bash \cite{bash} & 196 059 & dash \cite{dash} & 15 511 \\
        \hline
        util-linux \cite{util-linux} & 213 861 & ubase \cite{ubase} & 5 482 \\
        \hline
        coreutils \cite{coreutils} & 355 087 & \multirow{6}{*}{sbase \cite{sbase}} & \multirow{6}{*}{18 531} \\
        findutils \cite{findutils} & 184 414 & & \\
        sed \cite{sed} & 103 933 & & \\
        grep \cite{grep} & 131 031 & & \\
        tar \cite{tar} & 118 730 & & \\
        ed \cite{ed} & 3 095 & & \\
        \hline
        gawk \cite{gawk} & 226 032 & mawk \cite{mawk} & 20 112 \\
        \hline
        man-db \cite{man-db} & 163 880 & mandoc \cite{mandoc} & 64 590 \\
        \hline
        glibc \cite{glibc} & 1 495 641 & dietlibc \cite{dietlibc} & 88 075 \\
        \hline
        gcc \cite{gcc} & 8 526 557 & tcc \cite{tcc} & 81 481 \\
        \hline
    \end{tabular}
    \caption{Corresponding standard and selected packages with their sizes expressed in lines of code (LOC).}
\end{table}

\newpage

\subsubsection{Alternative packages}

In search of minimal software implementations the following packages\\ grouped by type were also considered.

\begin{table}[!ht]
    \centering
    \begin{tabular}{|c|c|c|}
        \hline
        Type & Package & Licence \\
        \hline
        \hline
        \multirow{5}{*}{Collection} & busybox \cite{busybox} & GPL-2.0 \\
        \cline{2-3}
        & toybox \cite{toybox} & 0BSD \\
        \cline{2-3}
        & heirloom \cite{heirloom} & Zlib et al. \footnotemark\\
        \cline{2-3}
        & 9base-6 \cite{9base} & MIT \\
        \cline{2-3}
        & s6-linux-utils \cite{s6-linux} & ISC \\
        \cline{2-3}
        & s6-portable-utils \cite{s6-portable} & ISC \\
        \hline
        \multirow{3}{*}{Bootloader} & lilo \cite{lilo} & BSD-3-Clause \\
        \cline{2-3}
        & refind \cite{refind} & GPL-3.0 \\
        \cline{2-3}
        & u-boot \cite{uboot} & BSD-3-Clause \\
        \hline
        \multirow{8}{*}{Init system} & s6 \cite{s6} & ISC \\
        \cline{2-3}
        & s6-rc \cite{s6-rc} & ISC \\
        \cline{2-3}
        & s6-linux-init \cite{s6-linux-init} & ISC \\
        \cline{2-3}
        & minit \cite{minit} & GPL-2.0 \\
        \cline{2-3}
        & openrc \cite{openrc} & BSD-2-Clause \\
        \cline{2-3}
        & runit \cite{runit} & BSD-3-Clause \\
        \cline{2-3}
        & shepherd \cite{shepherd} & GPL-3.0 \\
        \cline{2-3}
        & finit \cite{finit} & MIT \\
        \hline
        \multirow{3}{*}{Logger} & metalog \cite{metalog} & GPL-2.0 \\
        \cline{2-3}
        & rsyslog \cite{rsyslog} & GPL-3.0 \\
        \cline{2-3}
        & syslog-ng \cite{syslog-ng} & LGPL2.1 \\
        \hline
        \multirow{4}{*}{Shell} & mksh \cite{mksh} & ISC \\
        \cline{2-3}
        & oksh \cite{oksh} & Public Domain \\
        \cline{2-3}
        & rc \cite{rc} & Zlib \\
        \cline{2-3}
        & yash \cite{yash} & GPL-2.0 \\
        \hline
    \end{tabular}
    \caption{Alternative package types, names and corresponding licences.}
\end{table}

\footnotetext{Multiple FOSS and non-FOSS OSI open-source and FSF free software licences.}

\newpage

\begin{table}[!ht]
    \centering
    \begin{tabular}{|c|c|c|}
        \hline
        Type & Package & Licence \\
        \hline
        \hline
        \multirow{5}{*}{Text editor} & nano \cite{nano} & GPL-3.0 \\
        \cline{2-3}
        & \multirow{2}{*}{ex \cite{ex-vi}} & BSD-4-Clause \footnotemark \\
        & & and BSD-3-Clause \\
        \cline{2-3}
        & vim \cite{vim} & VIM \\
        \cline{2-3}
        & neovim \cite{neovim} & VIM and Apache-2.0 \\
        \hline
        Awk & oawk \cite{oawk} & ISC \\
        \hline
        \multirow{10}{*}{C Compiler} & 8cc \cite{8cc} & MIT \\
        \cline{2-3}
        & chibicc \cite{chibicc} & MIT \\
        \cline{2-3}
        & cparser \cite{cparser} & GPL-2.0 \\
        \cline{2-3}
        & c \cite{c} & BSD-2-Clause \\
        \cline{2-3}
        & \multirow{2}{*}{cproc \cite{cproc}} & ISC, MIT \\
        & & and Unlicense \\
        \cline{2-3}
        & kefir \cite{kefir} & GPL-3.0 \\
        \cline{2-3}
        & lacc \cite{lacc} & MIT \\
        \cline{2-3}
        & lcc \cite{lcc} & LCC \footnotemark \\
        \cline{2-3}
        & ack \cite{ack} & BSD-3-Clause \\
        \hline
        \multirow{2}{*}{Libc} & musl \cite{musl} & MIT \\
        \cline{2-3}
        & uClibc-ng \cite{uclibc-ng} & LGPLv2.1 \\
        \hline
    \end{tabular}
    \caption{Alternative package types, names and corresponding licences.}
\end{table}

\footnotetext[2]{Non OSI open-source free software licence.}
\footnotetext[3]{Non-FOSS open-source licence.}

\newpage
